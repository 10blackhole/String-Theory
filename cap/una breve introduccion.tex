\chapter{Una breve introducción}
\section{El camino hacia la unificación}
Cuatro fueras fundamentales han sido reconocidas en la naturaleza. Démosle un vistazo:
\begin{itemize}
    \item \textbf{La fuerza de gravedad}. Esta fuerza fue descubierta por Isaac Newton. La gravedad sufrió una profunda reformulación en la teoría de la Relatividad General de Albert Einstein. En dicha teoría, la arena espacio-temporal de la relatividad especial adquiere vida propia, y las fuerzas gravitacionales emergen de la curvatura de ese espacio-tiempo dinámico. La teoría de Einstein es una teoría clásica de la gravitación. No está formulada como una teoría cuántica.

    \item \textbf{La fuerza electromagnética}. Dicha fuerza es descrita por las ecuaciones de Maxwell. El electromagnetismo, o la teoría de Maxwell, es formulada como una teoría clásica de los campos electromagnéticos. Al contrario que la mecánica Newtoniana, la cuela es modificada por la Relatividad Especial, la teoría de Maxwell es completamente consistente con la Relativiad Especial.

    \item \textbf{La fuerza débil}. Esta fuerza es responsable de los procesos del decaimiento nuclear beta, en el cual un neutrino decae en un protón, un electrón y un antineutrino. En general, procesos que involucran neutrinos están medidos por fueras débiles. Mientras que el decaimiento nuclear beta ha sido conocido desde finales del siglo 19, el reconocimiento de que una nueva fuerza estaba en juego no fue hasta mediados del siglo 20. Las interacciones débiles son muchos más débiles que las interacciones electromagnéticas.

    \item \textbf{La fuerza fuerte}. Hoy en día llamada la fuerza de color. Esta fuerza mantiene juntos los constituyentes de neutrinos, protones, piones, y muchas otras partículas subnucleares. Estos constituyentes, llamados quarks, se mantiene tan apretados por la fuerza de color que ellos no se pueden ver isolados.
\end{itemize}

En los úñtimos 1960s el modelo de Wingberg-Salam para las interacciones \textbf{electrodébiles} juntó el electromagnetismo y la fuerza débil en un marco unificado. Fue necesario para una teoría de las interacciones débiles predictiva y consistente.

La teoría es inicialmente formulada con cuatro partículas sin masa que llevan las fuerzas. Un proceso de quiebre de la simetría les da masa a tres de estas partículas: los $W^+, W^-$ y los $Z^0$ . Estas partículas son las que llevan la fuerza débil. La partícula que queda sin masa es el fotón, que es el que lleva la fuerza electromagnética.

Las ecuaciones de Maxwell son ecuaciones del electromagnetismo clásico. Esta teoría no es ni aproximada ni una teoría correcta para fenómenos microscópicos. La electrodinámica cuántica (QED), la versión cuántica de el electrodinámica clásica, es requerida para calculos correctos en esta arena. En QED el fotón aparece como la cuanto del campo electromagnético. Las teoría de las interacciones débiles es también una teoría cuántica de partículas, entonces lo correcto, la teoría unificada es la teoría electrodébil cuántica.

El proceso de cuantización es también exitoso en el caso de la fuerza de color fuerte, en la teoría resultante es la cromodinámica cuántica (QCD). Los llevadores de la fuerza de color son ocho partículas sin masa. Estas son gluones coloridos, y como los quarks, ellos no pueden ser observados isolados.  Los quarks responden a los gluones porque ellos levan color. Los quarks pueden venir en tres colores.

En resúmen tenemos
\begin{equation}
    \text{Teoría electrodébil} + \text{QCD} = \text{Modelo Estándar de la física de partículas}
\end{equation}

En el Modelos Estándar (SM) hay algunas interacciones entre el sector electrodébil y el QCD porque algunas partículas sienten ambos tipos de fuerzas. El SM resume completamente e conocimiento actual de la física de partículas.

En el Modelo Estándar hay 12 llevadores de fuerza: los 8 gluones, los $W^-,W^+,Z^0$ y el fotón. Todos estos son \textbf{bosones}. También existen varias partículas de materia, todas ellas son \textbf{fermiones}. Las partículas de mateira son de dos tipos: \textbf{leptones}\footnote{Ver \url{https://es.wikipedia.org/wiki/Leptón}} y \textbf{quarks\footnote{Ver \url{https://es.wikipedia.org/wiki/Cuark}}}. Los leptones incluyen el electrón $e^-$, el muon $\mu^-$, el
